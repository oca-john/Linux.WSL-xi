% `LaTeX/TeX`文件保存时尽量使用英文,`TexStudio`或`Kile`调用`xeLaTeX/xeTeX`命令
% `xelatex -synctex=1 -interaction=nonstopmode LaTeX中文文件名.tex`,
% 由于编译器不支持中文字符编码,不能识别文件名中的中文,导致报错。
% 
% Asus-suse: Texlive + Kile/TexStudio
% Asus-win: Texlive + Kile/TexStudio
% Prob-arch: Texlive + Kile/TexStudio, 编译通过,并生成PDF。
% Prob-win: Texlive + Kile/TexStudio, 
% Expc-deepin: MikTex + Kile/TexWorks...
% Expc-win: MikTex + Kile/TexWorks, 编译通过,并生成PDF。


% 文档编码信息,用到的宏包
\documentclass[a4paper,10pt]{article}
\usepackage{xeCJK}					% 调用中文处理包,来自texlive-langchinese
\setCJKmainfont{FangSong}			% 全局CJK字体为仿宋
\setmainfont{Times New Roman}		% 全局字体为新罗马
% 已测试的字体有:
% 原生字体: FangSong, KaiTi, SimSun, SimHei, YouYuan, 
% 方正字体: FZYaoTi, FZShuTi, 
% 华文字体: STFangsong, STKaiti, STXingkai, STZhongsong, STXihei, STXinwei, STLiti


% Title信息,标题、作者、日期
\title{文章Title:LaTeX中英文混排不分栏模板}
\author{Oca John}
\date{2021年6月28日}


% 文档正文需要在"\begin{document}"和"\end{document}"之间
\begin{document}					% 显式声明正文开始
	\maketitle							% 正文中首先要生成title,依据是上面的信息
		
	% 摘要是正文的一部分,需要放在{document}内部
	% 摘要需要在"\begin{abstract}"和"\end{abstract}"之间
	\begin{abstract}
		
		基于什么背景问题,以及其产生的重大后果,什么样的处理是必要的(背景问题)。
		
		现有研究侧重于什么的研究为什么不能满足需求?本文提出的思路或方法如何切入,并解决该问题?
		
		核心的思想或方法是怎样的?具体以何种方案或策略执行?经过研究,得到的结果是怎样的?是否能解决之前提出的问题?
		
		综合而言,采用本文的策略或方法能够有效解决行业问题,使相关领域的问题得到优化或解决。
		
	\end{abstract}
	
	\section{第一部分:研究背景、相关研究概述、整理科学问题}
	
	对研究背景行业的宏观描述,以及本研究切入的背景问题是什么?
	
	相关研究有哪些?分别侧重于哪些研究点?
	
	立足于本文研究点,之前的研究无法回答或解决什么问题?依然存在的待研究的科学问题包括哪些?
	
	\section{第二部分:实验设计和主要解决思路,内在逻辑和主要方法}
	
	\subsection{实验设计与技术路线}
	
	这里是正文第一段落占位。
	
	\subsection{数据获取和预处理}
	
	\subsection{特征工程和模型构建}
	
	\subsection{模型训练、导出、部署}
	
	\subsection{模型测试、结果评估}
	
	\section{第三部分:结果与讨论}
	
\end{document}						% 显式声明正文结束
