% 由于`CTeX`包非常老旧,倾向于采用`XeCJK`实现中英文混排。
% Linux下可通过`texlive-langchinese`包安装`XeCJK`。
% 注意1:编译器仅限`XeLaTeX`,`LaTeX/TeX`文件保存时尽量使用英文,因为在`TexStudio`或`Kile`调用
% `xeLaTeX/xeTeX`的命令`xelatex -synctex=1 -interaction=nonstopmode LaTeX中文文件名.tex`时,
% 由于编译器不支持中文字符编码,可能无法识别文件名中的中文,而导致报错(仅系统默认不支持UTF8时)。
% 注意2:对默认支持的系统不要修改编码方式,此操作可能会导致其他文件或程序的乱码。
% 注意3:编辑器关闭启动画面(Kl),关闭实时预览(Kl),使用Xelatex编译器(Ts/Kl),Build方式改为compile方式(Kl),缩进方式改为空格(Ts/Kl)。
% 
% Asus-suse: Texlive + Kile/TexStudio, 编译通过,并生成PDF。
% Asus-win: Texlive + Kile/TexStudio, 编译通过,并生成PDF。
% Prob-arch: Texlive + Kile/TexStudio, 编译通过,并生成PDF[Arch未频繁更新,后续谨慎部署Arch Linux]。
% Prob-win: Texlive + Kile/TexStudio, 编译通过,并生成PDF[仅此平台无法识别中文文件名,采用修改系统编码的方式修正]。
% Expc-deepin: Texlive-Core + Kile, 编译通过,有报错,Deepin自带Okular不可卸载,不影响Kile使用,但影响系统依赖处理[后续不考虑不熟Deepin]。
% Expc-win: MikTex + Kile/TexWorks, 编译通过,并生成PDF。


% 文档编码信息,用到的宏包
\documentclass[a4paper,10pt]{article}
\usepackage{xeCJK}                  % 调用中文处理包,来自texlive-langchinese
\setCJKmainfont{FangSong}           % 全局CJK字体为仿宋
\setmainfont{Times New Roman}       % 全局字体为新罗马
% 已测试的字体有:
% 原生字体: FangSong, KaiTi, SimSun, SimHei, YouYuan, 
% 方正字体: FZYaoTi, FZShuTi, 
% 华文字体: STFangsong, STKaiti, STXingkai, STZhongsong, STXihei, STXinwei, STLiti


% Title信息,标题、作者、日期
\title{文章Title:LaTeX中英文混排不分栏模板}
\author{Oca John}
\date{2021年6月28日}


% 文档正文需要在"\begin{document}"和"\end{document}"之间
\begin{document}                    % 显式声明正文开始
    \maketitle                      % 正文中首先要生成title,依据是上面的信息
    
    % 摘要是正文的一部分,需要放在{document}内部
    % 摘要需要在"\begin{abstract}"和"\end{abstract}"之间
    \begin{abstract}
        
        基于什么背景问题,以及其产生的重大后果,什么样的处理是必要的(背景问题)。
        
        现有研究侧重于什么的研究为什么不能满足需求?本文提出的思路或方法如何切入,并解决该问题?
        
        核心的思想或方法是怎样的?具体以何种方案或策略执行?经过研究,得到的结果是怎样的?是否能解决之前提出的问题?
        
        综合而言,采用本文的策略或方法能够有效解决行业问题,使相关领域的问题得到优化或解决。
        
    \end{abstract}
    
    \section{第一部分:研究背景、相关研究概述、整理科学问题}
    
    对研究背景行业的宏观描述,以及本研究切入的背景问题是什么?
    
    相关研究有哪些?分别侧重于哪些研究点?
    
    立足于本文研究点,之前的研究无法回答或解决什么问题?依然存在的待研究的科学问题包括哪些?
    
    \section{第二部分:实验设计和主要解决思路,内在逻辑和主要方法}
    
    \subsection{实验设计与技术路线}
    
    这里是正文第一段落占位。
    
    \subsection{数据获取和预处理}
    
    \subsection{特征工程和模型构建}
    
    \subsection{模型训练、导出、部署}
    
    \subsection{模型测试、结果评估}
    
    \section{第三部分:结果与讨论}
    
\end{document}                      % 显式声明正文结束
